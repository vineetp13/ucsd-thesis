%
%
% UCSD Doctoral Dissertation Template
% -----------------------------------
% http://ucsd-thesis.googlecode.com
%
%


%% REQUIRED FIELDS -- Replace with the values appropriate to you

% No symbols, formulas, superscripts, or Greek letters are allowed
% in your title.
\title{Citizen-led Work using Social Computing and Procedural Guidance}

\author{Vineet Pandey}
\degreeyear{2019}

% Master's Degree theses will NOT be formatted properly with this file.
\degreetitle{Doctor of Philosophy}

\field{Computer Science}
%\specialization{Human-Computer Interaction}  % If you have a specialization, add it here

\chair{Professor Scott R. Klemmer}
% Uncomment the next line iff you have a Co-Chair
% \cochair{Professor Cochair Semimaster}
%
% Or, uncomment the next line iff you have two equal Co-Chairs.
%\cochairs{Professor Chair Masterish}{Professor Chair Masterish}

%  The rest of the committee members  must be alphabetized by last name.
\othermembers{
Professor James D. Hollan\\
Professor Rob Knight\\
Professor Donald A. Norman\\
Professor Laurel D. Riek\\
}
\numberofmembers{5} % |chair| + |cochair| + |othermembers|


%% START THE FRONTMATTER
%
\begin{frontmatter}

%% TITLE PAGES
%
%  This command generates the title, copyright, and signature pages.
%

\makefrontmatter

%% DEDICATION
%
%  You have three choices here:
%    1. Use the ``dedication'' environment.
%       Put in the text you want, and everything will be formated for
%       you. You'll get a perfectly respectable dedication page.
%
%
%    2. Use the ``mydedication'' environment.  If you don't like the
%       formatting of option 1, use this environment and format things
%       however you wish.
%
%    3. If you don't want a dedication, it's not required.
%
%
\begin{dedication}
  To the adventure of life
\end{dedication}


% \begin{mydedication} % You are responsible for formatting here.
%   \vspace{1in}
%   \begin{flushleft}
% 	To me.
%   \end{flushleft}
%
%   \vspace{2in}
%   \begin{center}
% 	And you.
%   \end{center}
%
%   \vspace{2in}
%   \begin{flushright}
% 	Which equals us.
%   \end{flushright}
% \end{mydedication}



%% EPIGRAPH
%
%  The same choices that applied to the dedication apply here.
%
\begin{epigraph} % The style file will position the text for you.

%I will give you a talisman. 
%\emph{Whenever you are in doubt, or when the self becomes too much with you, apply the following test. Recall the face of the poorest and the weakest man [woman] whom you may have seen, and ask yourself, if the step you contemplate is going to be of any use to him [her].}
\emph{Whenever you are in doubt, or when the self becomes too much with you, apply the following test. Recall the face of the poorest and the weakest person whom you may have seen, and ask yourself, if the step you contemplate is going to be of any use to them.}\\
Mahatma Gandhi

\vspace{10pt}

%  \emph{I don't do it for the 'Gram,\\
%	 I do it for Compton}\\
%Kendrick Lamar



%, ELEMENT., {\it DAMN}
\end{epigraph}

% \begin{myepigraph} % You position the text yourself.
%   \vfil
%   \begin{center}
%     {\bf Think! It ain't illegal yet.}
%
% 	\emph{---George Clinton}
%   \end{center}
% \end{myepigraph}


%% SETUP THE TABLE OF CONTENTS
%
\tableofcontents
\listoffigures  % Comment if you don't have any figures
%{% -- added by vineet to remove the word figure
%\let\oldnumberline\numberline%
%\renewcommand{\numberline}{\figurename~\oldnumberline}%
%\listoffigures%
%}
%\listoftables   % Comment if you don't have any tables



%% ACKNOWLEDGEMENTS
%
%  While technically optional, you probably have someone to thank.
%  Also, a paragraph acknowledging all coauthors and publishers (if
%  you have any) is required in the acknowledgements page and as the
%  last paragraph of text at the end of each respective chapter. See
%  the OGS Formatting Manual for more information.
%
\begin{acknowledgements}

%from gut instinct
%We thank all participants who used Gut Instinct and provided feedback. We thank members of Design Lab, especially Steven Dow and Derek Lomas, and Michael Bernstein for their useful comments on this work. We thank Brian Soe and Aliff Macapinlac for help developing the Gut Instinct website and running pilot studies. A Google Research Award and gift from SAP helped support this work.

%from docent
%We thank Docent participants for their feedback. We thank Chen Yang, Cody Doan, and Aliyah Clayton for help developing the website and running pilot studies. We thank Anupriya Tripathi and Nicolai Reeve for finding relevant scientific resources for the site, and Madeleine Ball for introducing Docent to the Open Humans community. A Google Research Award and gift from SAP helped support this work.

This dissertation deals with pragmatics; it’s only fair I start with those. 

Ideas are cheap, implementation is expensive. I want to thank NSF-IGE grant, Research Award from Google, and a gift from SAP for supporting this dissertation research. Summers spent elsewhere were supported by Microsoft Research and a European Research Council Grant. Special thanks to UCSD Computer Science Department for providing confirmed funding to students for the first year; this dissertation is a product of my broad explorations during that period. I’d like to think I haven’t been biased towards doing research (most of the times I did not know how I was funded) but I’ll let the reader take that call.

I express my sincere gratitude towards everyone who kept things running. Vanessa, Sara, Nina, Teenah, Ian, Olga, Clint, student assistants, and the broader Lab Ops team: your constant support made it much easier to focus on research. Thank you Jennifer, Emily, Alan, and Vidula for your seamless support for reimbursements and other tasks - both mundane and curious. Thank you, Michiko, for being so patient at one of the toughest tasks I know: managing Rob’s schedule. I’d love to see deep learning try that… Thank you Julie for the calm support you’ve provided me (and hundreds of PhD students) over the years. I salute the UCSD Human Research Protection Program (HRPP) for their critical support of my research.

*******
This dissertation brings together ideas from different universes. Scott Klemmer guided me through this process with both substance and style. When we first started working, Scott made his expectations clear: do important research. I’ve subsequently learned that success is a process; the outcome is just the last state. From Scott’s interaction with collaborators, I’ve learnt ways to develop common ground and make progress; from his feedback on prototypes, the importance of interaction design, and from his writing, how to write. I don’t consider principles for research to be vastly different from principles for life. Scott encouraged me to link the two; for instance, here are three principles I’ve added to my life after working with Scott: 1) look outwards, 2) do something, and 3) be empirical. Landing multiple postdoc opportunities is a result of Scott’s direct training and support.  -- work ethic

Working closely with Rob, I’ve been constantly inspired by his dedication and depth+breadth of knowledge. I always enjoyed the challenge of presenting to Rob; how much fat could I trim from my ideas and data to get Rob to think and provide useful suggestions. 

When reflecting on my life over a cup of chai, I’ll realize just how special my dissertation committee was. As an undergrad, I had read Don’s book and hated it because it blamed the designer for user’s mistakes. I’ve learnt my lesson. Jim’s positive disposition, willingness to share this time, and a rigorous work ethic always gave me confidence. If Jim could spend morning after morning working in the lab, so could I! Professor Laurel Riek’s excitement and probing questions about my research makes me even more excited for it! Thank you committee for the inspiration and the support. 

By definition, research prototypes are broken. I’m deeply indebted to all the students who’ve worked with me to make this less true. Guiding students made me more responsible, taught me new things (research and otherwise), and improved the odds of success for the platform. I hope they’ve gained as much from their experience as I have from working with them. Aliyah, Brian, Chen, Cody, Crystal, Dingmei, Hedy, Kaung, Liby, Orr, Rachel, Robert, Sam, Tushar, Senyan: I’m around in case you need any assistance.. 

Thanks are also due to my collaborators from the microbiome world -- Tomasz, Justine, Embriette, Daniel, and Amnon. 

***
To the few thousands of people who used my system: Thank you!  Thank you to community leaders and volunteers who piloted and used my platform; Mad, Bastian, Austin and Adriana, Ariel for testing out the platform. I also thank all the folks who’ve created all the software and libraries that this work uses. I would like to thank the anonymous CHI, CSCW, and UIST reviewers for their insightful comments; I’ve been lucky to receive long, thoughtful reviews. And even when I disagreed (fundamentally) with some concerns, I’ve learned that finally it’s on me to get the point across.

For the first half of my grad school journey, we were getting a group together. For the second half, I reaped the benefits. Thank you Ariel, Tricia, and Ailie for reading drafts, doing sushibooch(a), and just asking me to work a little more on that new joke. Misery loves company. Thank you Ailie for your flagrant generosity, your efforts towards meaningful causes, and cheerfulness. Thank you my dank dark small office that did not receive natural light in San Diego. Cse-misc was fun till they took conversations to other places for reasons only young minds understand. Thank you Steven Swanson for suggesting I pursue PhD in a topic I felt the most passionate about.

The Design Lab has supported me in living unfettered by disciplinary boundaries. Thank you Steven, Philip, Eric, Eli, and others for feedback and support. Thank you Michele for keeping the space running and for entertaining me - I tried paying it back. Thank you Derek, Stephanie, Seth for the walk and talk during the first year; those are some of my favourite memories. UCSD has some incredible creative minds.  and many others for broadening my horizon and showing me different ways to think.  Hillcrest friends -- thank you for the conversations.

Many deserve blame for introducing me to HCI research and then working alongside, and mentoring me. Laura (my first and last HCI TA): thank you for suggesting I might be decent at this HCI stuff, Chinmay: thank you for mentoring my first HCI project (was it too obvious I didn’t know what i was doing?), and Catherine: thank you for mentoring me with research design, with stats, and with mentoring undergraduates. Adam Rule was always there to take my stupid questions and somehow make them sound smart. The jump from systems to HCI is a big one (just ask James Mickens). After Social hour beers, Tianyin’s excitement about sysconfig ideas and our shared knowledge from NetApp helped. Yasmine patiently heard my ideas and provided useful feedback on our work. If you’ve worked with them and you still talk to them, keep them around. 

While I’ve always enjoyed doing useless things, my interest in research crystallized over multiple internships during undergrad. Thank you Sid Jaggi (CUHK), SS Rao (SNU), Bimal Roy (ISI Kolkata) -- you gave me appropriate supervision, inspiration, and positive support even when I had nothing to show. Most improtantly, you always treated me (and others) as people first. Sid proved to me that you could be a professor but be cool. Bimal Roy’s student centeredness and simplicity always humbled me. I owe a lot to my colleagues at NetApp Advanced Technology Group - it was my first time working on serious stuff and without Chhavi, Ranjit, Damaru, and Parag’s mentorship, I’d have flunked. Arvind Arasu, Krish Chatterjee hosted me and taught me. 

The great American philosopher Conan O Brian said, “it’s in the failure to become our idols that we find ourselves”. I’ve been inspired by many researchers whose name would be too much to fit here - some I’ve met, some know of my research, and some have said good things about my work (wheeee). we as HCI researchers have a grave responsibility; if it’s just about games and self interest, then there’s plenty of more fun games to be played out there. Thank you for your quest to do what’s important. I also thank all other folks outside of academia who do cool and imp stuff: this could be the education dept of jharkhand or the cancer experts at Tata memorial hospital -- where you make do with everything you have. 

**
This dissertation exists because many people have told me that they believe in me. At some mysterious point, their belief transfers to mine. *   1. family, school teachers, not my teachers at undergrad — most of them are incompetent buffoons whose only job was to crush the spirit of young men
2. all friends from Bariatu, DPS, Pilani, Bangalore, Hong Kong, Seoul, and my travel sojourns
    1. we don’t know which experience changes which things in our head — thank you for sharing a part of your life with me 

I still have more people to thank… phew. I want to thank my roommates, my friends from undergrad (WING, Gaonwaale, PCr, more), Karteek and Soumyadeep for listening to my complaints and sharing their grad school wisdom to guide this ship, grad school friends (varsha, radhe, skanda, saman, moein), tomight (Pushp, Pragya, AR, AT), and Arjun, Manish, Dimo, Marcela, Val, Niki and other funny people. I am grateful to meet so many friends from other places in grad school. Our differences pale in comparison to our similarities. My friends in San Diego (Martha, Lori)  deserve special applause for dealing with me in 4D.

San Diego is beyond special. A visiting friend exclaimed how I must have done great deeds in the previous life to live in SD. I agree with him. I’ve discovered so much in this place. and i thank people for great experiences. Thank you California for all the natural beauty. 

Cafe Influx -- thank you for the space that you’ve created. Inspiration comes from many places. Thank you to all the hardworking, creative artists who keep the rest of us going...

**
Through my research, I’ve learnt how to write, I’ve talked to personal heroes, learnt about how people learn, develop a better taste in many avenues, become far less bored in life cos there’s always something to observe, 

I learnt about the microbiome, circadian rhythms, sleep and more. And tweaking my lifestyle has been beneficial. I’m grateful to my trip across four countries---Turkey, Jordan, Egypt, Ethiopia ---which told me people had the same concerns and gave me the confusion and the confidence to jump ships.

We are living in times of mental crisis. I am lucky to have the support and insurance from the university to avail of these opportunities for over 50 sessions. In the process, I’ve learnt a lot more about myself, about the world, and about people. It has made me a better person.

My gratitude to all my teachers, friends, and strangers who’ve taught me things. I’m the luckiest person I know: things somehow work out when they have no business to do so.


i never knew these possibilities 

***
Culture creates the first design, the rest are just edits. Ma and pa: Thank you for showing me how to live a life true to values, for taking care of me when I was incapable of doing so, and for the freedom to do things I did not understand and could not explain. 


Even in its most miserable moments, I’ve fundamentally believed that graduate school is a privilege for me. Almost no one gets to live this life of single-minded pursuit of knowledge, passion, next big idea - whatever it be for you. Why? I hope we can change that. 

Life is an adventure; it must be lived like one. 

 




\vspace{0.25in}

\textsc{Chapter 3}, in part, includes portions of material as it appears in \emph{Gut Instinct: Creating Scientific Theories with Online Learners} by Vineet Pandey, Amnon Amir, Justine W. Debelius, Embriette R. Hyde, Tomasz Kosciolek, Rob Knight, and Scott R. Klemmer in the proceedings of the ACM Conference on Human Factors in Computing Systems (CHI 2017). The dissertation author was the primary investigator and author of this paper.

\textsc{Chapter 4}, in part, includes portions of material as it appears in \emph{Docent: Transforming Personal Intuitions to Scientific Hypotheses through Content Learning and Process Training} by Vineet Pandey, Justine W. Debelius, Embriette R. Hyde, Tomasz Kosciolek, Rob Knight, and Scott R. Klemmer in the proceedings of the ACM Conference on Learning at Scale (L@S 2018). The dissertation author was the primary investigator and author of this paper.

\textsc{Chapter 5}, in part, includes portions of material as it appears in the paper under preparation \emph{Galileo: Procedural Support for Citizen Experimentation} by Vineet Pandey, Tushar Koul, Chen Yang, Daniel McDonald, Mad Price Ball, Bastian Greshake Tzovaras, Rob Knight, and Scott R. Klemmer. The dissertation author was the primary investigator and author of this paper.

\end{acknowledgements}

%% VITA
%
%  A brief vita is required in a doctoral thesis. See the OGS
%  Formatting Manual for more information.
%
\begin{vitapage}
\begin{vita}
%  \item[2011] B.~Engineering. in Computer Science \emph{cum laude}, Birla Institute of Technology \& Science, Pilani, India
  \item[2011] B.~Engineering in Computer Science, \\Birla Institute of Technology \& Science, Pilani, India
  \item[2016] M.S. in Computer Science, University of California San Diego
  \item[2019] Ph.~D. in Computer Science, University of California San Diego
\end{vita}
\begin{publications}
%  \item Your Name, ``A Simple Proof Of The Riemann Hypothesis'', \emph{Annals of Math}, 314, 2007.
%  \item Your Name, Euclid, ``There Are Lots Of Prime Numbers'', \emph{Journal of Primes}, 1, 300 B.C.
 \item Vineet Pandey, Amnon Amir, Justine W. Debelius, Embriette R. Hyde, Tomasz Kosciolek, Rob Knight, Scott R. Klemmer. ``Gut Instinct: Creating Scientific Theories
with Online Learners'', \emph{ACM Conference on Human Factors in Computing Systems (CHI)}, 6825-6836, 2017.
\item Vineet Pandey, Justine W. Debelius, Embriette R. Hyde, Tomasz Kosciolek, Rob Knight, Scott R. Klemmer. ``Docent: Transforming Personal Intuitions to Scientific Hypotheses
through Content Learning and Process Training'', \emph{ACM Learning at Scale}, 9:1-9:10,  2018.
\item Vineet Pandey, Tushar Koul, Chen Yang, Daniel McDonald, Mad Price Ball, Bastian Greshake Tzovaras, Rob Knight, Scott R. Klemmer. ``Galileo: Procedural Support for Citizen Experimentation'', \emph{In Preparation},  2019.

\end{publications}
\end{vitapage}


%% ABSTRACT
%
%  Doctoral dissertation abstracts should not exceed 350 words.
%   The abstract may continue to a second page if necessary.
%
\begin{abstract}
Online platforms enable people to interact with friends, family, and the world at large. How might people go beyond sharing stories and ideas to building and testing theories in the real world? While many are motivated to dig deeper into their lived experience, limited expertise and lack of platform support make complex activities like experimentation dauntingly hard. Novices benefit greatly from expert guidance: this thesis advocates baking the guidance into the interface itself.

This dissertation introduces \textit{procedural guidance} to build just-in-time expertise for difficult tasks. Procedural guidance has multiple advantages: it is minimal, leverages teachable moments, and can be ability-specific. This dissertation instantiates this insight of procedural guidance through a sequence of increasingly complex social computing systems: \textit{Gut Instinct} for curating ideas,  \textit{Docent} for generating hypotheses, and \textit{Galileo} for citizen-led experiments.

\textit{Gut Instinct} hosts online learning materials and enables people to collaboratively brainstorm potential influences on people’s microbiome. \textit{Docent} explicitly teaches people to create hypotheses by combining personal insights and online learning with task-specific scaffolding. Finally, \textit{Galileo} reifies experimentation in the software, provides multiple roles for contribution, and automatically manages interdependencies. Multiple evaluations—controlled experiments and field deployments with online communities including American Gut participants—demonstrate that procedural guidance enables people to transform intuitions to hypotheses and structurally-sound experiments. By enabling people to draw on lived experience, this dissertation harbingers a future where people can convert their intuitions to actionable plans and implement these plans with online communities. This dissertation concludes by discussing opportunities for complex work using social computing platforms.


\end{abstract}


\end{frontmatter}
