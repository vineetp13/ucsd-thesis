%
%
% UCSD Doctoral Dissertation Template
% -----------------------------------
% http://ucsd-thesis.googlecode.com
%
%


%% REQUIRED FIELDS -- Replace with the values appropriate to you

% No symbols, formulas, superscripts, or Greek letters are allowed
% in your title.
\title{Citizen-led Work using Social Computing and Procedural Guidance}

\author{Vineet Pandey}
\degreeyear{2019}

% Master's Degree theses will NOT be formatted properly with this file.
\degreetitle{Doctor of Philosophy}

\field{Computer Science}
%\specialization{Human-Computer Interaction}  % If you have a specialization, add it here

\chair{Professor Scott R. Klemmer}
% Uncomment the next line iff you have a Co-Chair
% \cochair{Professor Cochair Semimaster}
%
% Or, uncomment the next line iff you have two equal Co-Chairs.
%\cochairs{Professor Chair Masterish}{Professor Chair Masterish}

%  The rest of the committee members  must be alphabetized by last name.
\othermembers{
Professor James D. Hollan\\
Professor Rob Knight\\
Professor Donald A. Norman\\
Professor Laurel D. Riek\\
}
\numberofmembers{5} % |chair| + |cochair| + |othermembers|


%% START THE FRONTMATTER
%
\begin{frontmatter}

%% TITLE PAGES
%
%  This command generates the title, copyright, and signature pages.
%

\makefrontmatter

%% DEDICATION
%
%  You have three choices here:
%    1. Use the ``dedication'' environment.
%       Put in the text you want, and everything will be formated for
%       you. You'll get a perfectly respectable dedication page.
%
%
%    2. Use the ``mydedication'' environment.  If you don't like the
%       formatting of option 1, use this environment and format things
%       however you wish.
%
%    3. If you don't want a dedication, it's not required.
%
%
\begin{dedication}
  To the adventure of life
\end{dedication}


% \begin{mydedication} % You are responsible for formatting here.
%   \vspace{1in}
%   \begin{flushleft}
% 	To me.
%   \end{flushleft}
%
%   \vspace{2in}
%   \begin{center}
% 	And you.
%   \end{center}
%
%   \vspace{2in}
%   \begin{flushright}
% 	Which equals us.
%   \end{flushright}
% \end{mydedication}



%% EPIGRAPH
%
%  The same choices that applied to the dedication apply here.
%
\begin{epigraph} % The style file will position the text for you.

%I will give you a talisman. 
%\emph{Whenever you are in doubt, or when the self becomes too much with you, apply the following test. Recall the face of the poorest and the weakest man [woman] whom you may have seen, and ask yourself, if the step you contemplate is going to be of any use to him [her].}
\emph{Whenever you are in doubt, or when the self becomes too much with you, apply the following test. Recall the face of the poorest and the weakest person whom you may have seen, and ask yourself, if the step you contemplate is going to be of any use to them.}\\
Mahatma Gandhi

\vspace{10pt}

%  \emph{I don't do it for the 'Gram,\\
%	 I do it for Compton}\\
%Kendrick Lamar



%, ELEMENT., {\it DAMN}
\end{epigraph}

% \begin{myepigraph} % You position the text yourself.
%   \vfil
%   \begin{center}
%     {\bf Think! It ain't illegal yet.}
%
% 	\emph{---George Clinton}
%   \end{center}
% \end{myepigraph}


%% SETUP THE TABLE OF CONTENTS
%
\tableofcontents
\listoffigures  % Comment if you don't have any figures
%{% -- added by vineet to remove the word figure
%\let\oldnumberline\numberline%
%\renewcommand{\numberline}{\figurename~\oldnumberline}%
%\listoffigures%
%}
%\listoftables   % Comment if you don't have any tables



%% ACKNOWLEDGEMENTS
%
%  While technically optional, you probably have someone to thank.
%  Also, a paragraph acknowledging all coauthors and publishers (if
%  you have any) is required in the acknowledgements page and as the
%  last paragraph of text at the end of each respective chapter. See
%  the OGS Formatting Manual for more information.
%
\begin{acknowledgements}
This dissertation has been a long, exciting journey of forming questions, finding answers, and just figuring things out. Along this terrific journey, I have been lucky to enjoy the support of many people and institutions. These words of acknowledgement are just some thoughts presented in a rushed form; my gratitude is way broader and deeper…

Ideas are cheap, implementation is expensive. I want to thank NSF-IGE grant, a Google Research Award, and a gift from SAP for supporting this dissertation research. Summers spent elsewhere were supported by Microsoft Research and a European Research Council Grant. Thank you UCSD Computer Science department for providing confirmed funding to students for the first year; this dissertation is a product of my broad explorations during that period. I express my sincere gratitude to everyone who keeps \textit{things} running. Vanessa, Sara, Nina, Teenah, Ian, Olga, Clint, student assistants, and the broader Lab Ops team: your constant support made it easier to focus on research. Thank you Jennifer, Emily, Alan, and Vidula for your seamless support with reimbursements and more. Thank you, Michiko, for being so patient at one of the toughest tasks I know: managing Rob’s schedule. Thank you Julie for the calm support you’ve provided me (and hundreds of CSE PhD students) over the years. I salute the UCSD Human Research Protection Program (HRPP) for their critical support of this and other research. 

This dissertation brings together ideas from different universes. Scott Klemmer guided me through this process with both substance and style. When we first started working together, Scott made his expectations clear: do. important. research. Over the years, I’ve learned that success is a daily grind; \textit{overnight} successes take years in the fermentation jar. If you’ve enjoyed using this dissertation’s systems or reading the papers, you have also gained from Scott’s wisdom about research, his feedback on my writing, and his interaction design sorcery. With Scott’s encouragement, I’ve transferred many principles between research and life; here are three: 1) look outwards, 2) do something, and 3) be empirical.

Years from now, I'll reflect on life and realize just how special my dissertation committee was. At every meeting, Rob inspired me with his dedication, knowledge (of microbiome and beyond), and argumentation. Back in Bangalore, I loathed Don’s book: it blamed the designer for users’ mistakes. I stand corrected. Don’s push to \textit{start with people} has been life-changing for this engineer-turned-researcher. Jim’s positive disposition, willingness to share his time, and a high tolerance for puns gifted me with a calm confidence. And if Jim could spend his mornings working in the lab, so could I! Professor Laurel Riek’s excitement and probing questions about my research have both motivated and tested me. Thank you committee for the inspiration and the support. 

Guiding students made me more responsible and taught me new techniques. Aliyah, Brian, Chen, Cody, Crystal, Dingmei, Hedy, Kaung, Liby, Orr, Rachel, Robert, Senyan, Tushar: Thank you and I hope I have been helpful. I’m around. Microbiome experts---Tomasz, Justine, Embriette, Daniel, and Amnon---helped me through the challenges of doing careful science. Community leaders and volunteers---Adriana, Ariel, Austin, Bastian, Mad---piloted and used my platform. Many deserve blame for introducing me to HCI research: Laura (my first and last HCI TA) suggested I might be decent at this stuff; Chinmay mentored my first project about learning and social computing; and Catherine introduced me to research design. Adam (Rule) is the oracle who transforms my stupid questions to coherent ideas. The jump from systems to HCI research was big, scary, and exciting; Tianyin’s passion for sysconfig ideas eased the transition. Early on, Yasmine patiently showed me the ropes of MOOC research. UCSD has some incredibly creative minds; being in their vicinity is reason enough to try harder. My interest in research crystallized over multiple internships during undergrad. Sid Jaggi (Chinese University of Hong Kong), SS Rao (Seoul National University), and Bimal Roy (Indian Statistical Institute, Kolkata) supervised, inspired, and supported me even when I had little to show for my research efforts.  Arvind Arasu and Krish Chatterjee were always around to discuss ideas when they hosted me over summers during grad school. To all my collaborators, mentors, and the few thousands of people who tried this dissertation's research systems: \textit{Thank you}! 

The Design Lab has supported me in thinking unfettered by disciplinary boundaries. Thank you Eli, Eric, Lilly, Philip, Steven for feedback and alternate takes on my research. Thank you Michele for the banter and laughs. Thank you Ariel, Tricia, Ailie, crewtoms for providing feedback, sharing our grad school challenges, and for doing sushibooch(a). Ailie's flagrant generosity has benefitted me over many years now: my drafts would have the many more errors otherwise. You have been the light in our tiny, dank office. Research chats around town w/Derek, Seth, and the Shermanator are some of my favourite memories. I’ve received many thoughtful reviews from CHI, CSCW, and UIST reviewers. I deeply appreciate your honest efforts; let’s make great reviews the norm.

What’s life without friends? Karteek and Soumyadeep listened to my complaints and shared their grad school wisdom. Vicky and Eduardo dealt with messy kitchen sinks, undergrad friends---WING, Gaonwaale, PCr, more--- made fun of me, and grad school friends---Varsha, Radhe, Skanda, Saman, Moein, Pushp, Pragya, Rangmaster, Anupriya, Arjun, John, Manish, Mario, Dimo, Marcela, Akshay, Val, Tateda, Niki---kept things interesting with movies, food, music, and more. Our differences pale in comparison to our similarities. San Diego is a special place; its residents Martha and Lori deserve special applause for dealing with this tired grad student and introducing him to special places like Barrio Logan and OB. Through counseling opportunities offered by the university and supported by insurance, I’ve learnt plenty about myself. I hope for a better future with less mental health concerns and greater resources to work with them.

Culture creates the first design, the rest are just edits. My deepest regards are reserved for my family. Thank you for living in ways many find worthy of emulating. You've blessed me with the freedom to do things I did not understand and could not explain. As they say in California: \textit{Namaste}. 



\vspace{0.25in}

\textsc{Chapter 3}, in part, includes portions of material as it appears in \emph{Gut Instinct: Creating Scientific Theories with Online Learners} by Vineet Pandey, Amnon Amir, Justine W. Debelius, Embriette R. Hyde, Tomasz Kosciolek, Rob Knight, and Scott R. Klemmer in the proceedings of the ACM Conference on Human Factors in Computing Systems (CHI 2017). The dissertation author was the primary investigator and author of this paper.

\textsc{Chapter 4}, in part, includes portions of material as it appears in \emph{Docent: Transforming Personal Intuitions to Scientific Hypotheses through Content Learning and Process Training} by Vineet Pandey, Justine W. Debelius, Embriette R. Hyde, Tomasz Kosciolek, Rob Knight, and Scott R. Klemmer in the proceedings of the ACM Conference on Learning at Scale (L@S 2018). The dissertation author was the primary investigator and author of this paper.

\textsc{Chapter 5}, in part, includes portions of material as it appears in the paper under preparation \emph{Galileo: Procedural Support for Citizen Experimentation} by Vineet Pandey, Tushar Koul, Chen Yang, Daniel McDonald, Mad Price Ball, Bastian Greshake Tzovaras, Rob Knight, and Scott R. Klemmer. The dissertation author was the primary investigator and author of this paper.

\end{acknowledgements}

%% VITA
%
%  A brief vita is required in a doctoral thesis. See the OGS
%  Formatting Manual for more information.
%
\begin{vitapage}
\begin{vita}
%  \item[2011] B.~Engineering. in Computer Science \emph{cum laude}, Birla Institute of Technology \& Science, Pilani, India
  \item[2011] B.~Engineering in Computer Science, \\Birla Institute of Technology \& Science, Pilani, India
  \item[2016] M.S. in Computer Science, University of California San Diego
  \item[2019] Ph.~D. in Computer Science, University of California San Diego
\end{vita}
\begin{publications}
%  \item Your Name, ``A Simple Proof Of The Riemann Hypothesis'', \emph{Annals of Math}, 314, 2007.
%  \item Your Name, Euclid, ``There Are Lots Of Prime Numbers'', \emph{Journal of Primes}, 1, 300 B.C.
 \item Vineet Pandey, Amnon Amir, Justine W. Debelius, Embriette R. Hyde, Tomasz Kosciolek, Rob Knight, Scott R. Klemmer. ``Gut Instinct: Creating Scientific Theories
with Online Learners'', \emph{ACM Conference on Human Factors in Computing Systems (CHI)}, 6825-6836, 2017.
\item Vineet Pandey, Justine W. Debelius, Embriette R. Hyde, Tomasz Kosciolek, Rob Knight, Scott R. Klemmer. ``Docent: Transforming Personal Intuitions to Scientific Hypotheses
through Content Learning and Process Training'', \emph{ACM Learning at Scale}, 9:1-9:10,  2018.
\item Vineet Pandey, Tushar Koul, Chen Yang, Daniel McDonald, Mad Price Ball, Bastian Greshake Tzovaras, Rob Knight, Scott R. Klemmer. ``Galileo: Procedural Support for Citizen Experimentation'', \emph{In Preparation},  2019.

\end{publications}
\end{vitapage}


%% ABSTRACT
%
%  Doctoral dissertation abstracts should not exceed 350 words.
%   The abstract may continue to a second page if necessary.
%
\begin{abstract}
Online platforms enable people to interact with friends, family, and the world at large. How might people go beyond sharing stories and ideas to building and testing theories in the real world? While many are motivated to dig deeper into their lived experience, limited expertise and lack of platform support make complex activities like experimentation dauntingly hard. Novices benefit greatly from expert guidance: this thesis advocates baking the guidance into the interface itself.

This dissertation introduces \textit{procedural guidance} to build just-in-time expertise for difficult tasks. Procedural guidance has multiple advantages: it is minimal, leverages teachable moments, and can be ability-specific. This dissertation instantiates this insight of procedural guidance through a sequence of increasingly complex social computing systems: \textit{Gut Instinct} for curating ideas,  \textit{Docent} for generating hypotheses, and \textit{Galileo} for citizen-led experiments.

\textit{Gut Instinct} hosts online learning materials and enables people to collaboratively brainstorm potential influences on people’s microbiome. \textit{Docent} explicitly teaches people to create hypotheses by combining personal insights and online learning with task-specific scaffolding. Finally, \textit{Galileo} reifies experimentation in the software, provides multiple roles for contribution, and automatically manages interdependencies. Multiple evaluations—controlled experiments and field deployments with online communities including American Gut participants—demonstrate that procedural guidance enables people to transform intuitions to hypotheses and structurally-sound experiments. By enabling people to draw on lived experience, this dissertation harbingers a future where people can convert their intuitions to actionable plans and implement these plans with online communities. This dissertation concludes by discussing opportunities for complex work using social computing platforms.


\end{abstract}


\end{frontmatter}
