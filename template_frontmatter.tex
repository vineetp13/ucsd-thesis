%
%
% UCSD Doctoral Dissertation Template
% -----------------------------------
% http://ucsd-thesis.googlecode.com
%
%


%% REQUIRED FIELDS -- Replace with the values appropriate to you

% No symbols, formulas, superscripts, or Greek letters are allowed
% in your title.
\title{Social Computing with Procedural Guidance}

\author{Vineet Pandey}
\degreeyear{2019}

% Master's Degree theses will NOT be formatted properly with this file.
\degreetitle{Doctor of Philosophy}

\field{Computer Science}
%\specialization{Human-Computer Interaction}  % If you have a specialization, add it here

\chair{Professor Scott R. Klemmer}
% Uncomment the next line iff you have a Co-Chair
% \cochair{Professor Cochair Semimaster}
%
% Or, uncomment the next line iff you have two equal Co-Chairs.
%\cochairs{Professor Chair Masterish}{Professor Chair Masterish}

%  The rest of the committee members  must be alphabetized by last name.
\othermembers{
Professor James D. Hollan\\
Professor Rob Knight\\
Professor Donald A. Norman\\
Professor Laurel D. Riek\\
}
\numberofmembers{5} % |chair| + |cochair| + |othermembers|


%% START THE FRONTMATTER
%
\begin{frontmatter}

%% TITLE PAGES
%
%  This command generates the title, copyright, and signature pages.
%

\makefrontmatter

%% DEDICATION
%
%  You have three choices here:
%    1. Use the ``dedication'' environment.
%       Put in the text you want, and everything will be formated for
%       you. You'll get a perfectly respectable dedication page.
%
%
%    2. Use the ``mydedication'' environment.  If you don't like the
%       formatting of option 1, use this environment and format things
%       however you wish.
%
%    3. If you don't want a dedication, it's not required.
%
%
\begin{dedication}
  To the adventure of life
\end{dedication}


% \begin{mydedication} % You are responsible for formatting here.
%   \vspace{1in}
%   \begin{flushleft}
% 	To me.
%   \end{flushleft}
%
%   \vspace{2in}
%   \begin{center}
% 	And you.
%   \end{center}
%
%   \vspace{2in}
%   \begin{flushright}
% 	Which equals us.
%   \end{flushright}
% \end{mydedication}



%% EPIGRAPH
%
%  The same choices that applied to the dedication apply here.
%
\begin{epigraph} % The style file will position the text for you.

%I will give you a talisman. 
%\emph{Whenever you are in doubt, or when the self becomes too much with you, apply the following test. Recall the face of the poorest and the weakest man [woman] whom you may have seen, and ask yourself, if the step you contemplate is going to be of any use to him [her].}
\emph{Whenever you are in doubt, or when the self becomes too much with you, apply the following test. Recall the face of the poorest and the weakest person whom you may have seen, and ask yourself, if the step you contemplate is going to be of any use to them.}\\
Mahatma Gandhi

\vspace{10pt}

%  \emph{I don't do it for the 'Gram,\\
%	 I do it for Compton}\\
%Kendrick Lamar



%, ELEMENT., {\it DAMN}
\end{epigraph}

% \begin{myepigraph} % You position the text yourself.
%   \vfil
%   \begin{center}
%     {\bf Think! It ain't illegal yet.}
%
% 	\emph{---George Clinton}
%   \end{center}
% \end{myepigraph}


%% SETUP THE TABLE OF CONTENTS
%
\tableofcontents
\listoffigures  % Comment if you don't have any figures
\listoftables   % Comment if you don't have any tables



%% ACKNOWLEDGEMENTS
%
%  While technically optional, you probably have someone to thank.
%  Also, a paragraph acknowledging all coauthors and publishers (if
%  you have any) is required in the acknowledgements page and as the
%  last paragraph of text at the end of each respective chapter. See
%  the OGS Formatting Manual for more information.
%
\begin{acknowledgements}

%from gut instinct
%We thank all participants who used Gut Instinct and provided feedback. We thank members of Design Lab, especially Steven Dow and Derek Lomas, and Michael Bernstein for their useful comments on this work. We thank Brian Soe and Aliff Macapinlac for help developing the Gut Instinct website and running pilot studies. A Google Research Award and gift from SAP helped support this work.

%from docent
%We thank Docent participants for their feedback. We thank Chen Yang, Cody Doan, and Aliyah Clayton for help developing the website and running pilot studies. We thank Anupriya Tripathi and Nicolai Reeve for finding relevant scientific resources for the site, and Madeleine Ball for introducing Docent to the Open Humans community. A Google Research Award and gift from SAP helped support this work.

To populate

\vspace{0.25in}

\textsc{Chapter 3}, in part, includes portions of material as it appears in \emph{Gut Instinct: Creating Scientific Theories with Online Learners} by Vineet Pandey, Amnon Amir, Justine W. Debelius, Embriette R. Hyde, Tomasz Kosciolek, Rob Knight, and Scott R. Klemmer in the proceedings of the ACM Conference on Human Factors in Computing Systems (CHI 2017). The dissertation author was the primary investigator and author of this paper.

\textsc{Chapter 4}, in part, includes portions of material as it appears in \emph{Docent: Transforming Personal Intuitions to Scientific Hypotheses through Content Learning and Process Training} by Vineet Pandey, Justine W. Debelius, Embriette R. Hyde, Tomasz Kosciolek, Rob Knight, and Scott R. Klemmer in the proceedings of the ACM Conference on Learning at Scale (L@S 2018). The dissertation author was the primary investigator and author of this paper.

\textsc{Chapter 5}, in part, includes portions of material as it appears in  the submitted paper \emph{Galileo: Procedural Support for Citizen Experimentation} by Vineet Pandey, Tushar Koul, Chen Yang, Daniel McDonald, Mad Price Ball, Bastian Greshake Tzovaras, Rob Knight, and Scott R. Klemmer. The dissertation author was the primary investigator and author of this paper.

\end{acknowledgements}

%% VITA
%
%  A brief vita is required in a doctoral thesis. See the OGS
%  Formatting Manual for more information.
%
\begin{vitapage}
\begin{vita}
%  \item[2011] B.~Engineering. in Computer Science \emph{cum laude}, Birla Institute of Technology \& Science, Pilani, India
  \item[2011] B.~Engineering in Computer Science, \\Birla Institute of Technology \& Science, Pilani, India
  \item[2016] M.S. in Computer Science, University of California San Diego
  \item[2019] Ph.~D. in Computer Science, University of California San Diego
\end{vita}
\begin{publications}
%  \item Your Name, ``A Simple Proof Of The Riemann Hypothesis'', \emph{Annals of Math}, 314, 2007.
%  \item Your Name, Euclid, ``There Are Lots Of Prime Numbers'', \emph{Journal of Primes}, 1, 300 B.C.
 \item Vineet Pandey, Amnon Amir, Justine W. Debelius, Embriette R. Hyde, Tomasz Kosciolek, Rob Knight, Scott R. Klemmer. ``Gut Instinct: Creating Scientific Theories
with Online Learners'', \emph{ACM Conference on Human Factors in Computing Systems (CHI)}, 6825-6836, 2017.
\item Vineet Pandey, Justine W. Debelius, Embriette R. Hyde, Tomasz Kosciolek, Rob Knight, Scott R. Klemmer. ``Docent: Transforming Personal Intuitions to Scientific Hypotheses
through Content Learning and Process Training'', \emph{ACM Learning at Scale}, 9:1-9:10,  2018.
\item Vineet Pandey, Tushar Koul, Chen Yang, Daniel McDonald, Mad Price Ball, Bastian Greshake Tzovaras, Rob Knight, Scott R. Klemmer. ``Galileo: Procedural Support for Citizen Experimentation'', \emph{In Submission},  2019.

\end{publications}
\end{vitapage}


%% ABSTRACT
%
%  Doctoral dissertation abstracts should not exceed 350 words.
%   The abstract may continue to a second page if necessary.
%
\begin{abstract}
Online social platforms provide a core infrastructure for people to interact with friends, family, and the world at large. People share stories, data, and ideas from their lived experience on topics of personal interest such as health and lifestyle. How might people go beyond sharing anecdotes to building and testing theories in the real world? While many are motivated to dig deeper to answer their questions, limited prior expertise and lack of support from social platforms make it near impossible to perform complex activities like experimentation. This situation motivates developing new platforms that enable people perform complex work by developing expertise and sourcing from their communities.

Traditional approaches for collective work use experts to guide novices; this thesis advocates a novel approach: bake in the guidance in the interface itself. This dissertation presents one general idea — \textit{procedural guidance} — to build just-in-time expertise for people to leverage their lived experience to perform tasks that they cannot do otherwise. Procedural guidance has multiple advantages: it is minimal, leverages teachable moments, and can be ability-specific. This dissertation instantiates this insight of procedural guidance through a sequence of increasingly complex social computing systems: Gut Instinct for curating ideas, Docent for generating hypotheses, and Galileo for citizen-led experiments. 

Gut Instinct hosts online learning materials and enables people to collaboratively brainstorm potential influences on people’s microbiome. Docent explicitly teaches people to create hypotheses by combining personal insights and online learning with task-specific scaffolding. Finally, Galileo reifies experimentation in the software, provides multiple roles for contribution, and automatically manages interdependencies. Multiple evaluations—including between-subjects experiments and field deployment with groups including American Gut (world’s largest citizen science project)—demonstrate that procedural guidance enables people to share their intuitions, transform these intuitions into hypotheses, and test such hypotheses with experiments. 
%Workflows, rubrics, templates, and automated support provide perform complex work in the \textit{absence of experts}. 

By introducing guidance in social computing for complex work, this dissertation provides one approach to distribute expertise around the globe using online systems. By enabling people to get started, make progress, and ask others, this dissertation harbingers a future where people can go beyond sharing their ideas to convert them to specific actionable plans and implement them with their communities. Finally, this dissertation hopes to develop social computing as a platform not just for sharing ideas but to co-ordinate action on complex activities that otherwise require expertise. Finally, this dissertation concludes by discussing future directions in enabling complex work using social computing platforms.

\end{abstract}


\end{frontmatter}
