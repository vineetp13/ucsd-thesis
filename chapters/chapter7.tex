
%%%%%%%%%%%%%%%%%%%%%%%%%%%%%%%%%
\chapter{Conclusion}
%"build theory, think expansively, and tackle broader conceptual issues, not (primarily) to reiterate the literal outcomes of your work.”-srk

Themes
\begin{itemize}
\item 0. improve idea
\item 1. improve techniques  -- how to teach better and more complex stuff
\item 2. improve systems -- how to bake these ideas in the interface and backend
\item 3. improve results -- how to scale beyond fun experiments
\item 4. implications for the individual, the community, and machines -- challenges, potential ways, etc... -- atoms, bits, culture
\end{itemize}

\section{Shortcomings / Challenges / limitations}
%"What were the challenges of this work? For example, the difficulty of gathering longitudinal data b/c people are tied to their email client. How would you recommend others address these challenges?" - srk
\begin{itemize}
\item is this really scientific work? what’s missing?
1. place-controlled blind exp \\
2. self-administration \\
3. difficulty of manipulation consistency 
---  causality (and ML)   
\item Use exsiting corpus of procedural learning -- wikihow provides a corpus -- how does this interface wiht expert patterns ideas
\item Engagement is hard man -- see msb write up -- show numbers of dropout etc..
\item worsened by making people work in lockstep --  difficulty of implementing on the ground - akshay roongta thesis -- Cultural effects in self organization of crowds?
\item hard to measure learning gains -- do people even learn?
\item dual objective systems

\end{itemize}


data use
1. what if data loss happens — what if ppl are tagged with conditions 




%%%%%%
\section{Implications}
%"If the world were to transform to rely heavily on your work, what broader technical and societal transformations would arise? How would your work scale? What would the social and technical challenges be? This might be a good place to connect your work to broader (relevant) writing on the role of technology in society"-srk
\begin{itemize}
\item personal intuitions provide one set of hypo generation -- others could be folk theories, data tracked from devices, observations, playing around with data using Vega-lite; ways to evaluate could be different too -- data analysiss
\item Also, what really makes you a scientist? -- how to enable deeper work in microbiome
\item Polio took such a long time from results to policy --  how to get citizens to create such knowledge -- link to health world
\end{itemize}


%%%%%%
\section{Systems/domains} 
%\section{New Systems? Other areas of application?}
% "where else could your techniques and insights apply? Think big."-srk

\begin{itemize}
\item We found that our approach works well for scineitifc experimentation; what about toher types of work and genre?
--talk about my procedural rules for writing papers or doing data science or activism \\
-- the challenge: proving that many other classes of work also fall in this category
--interpreting work around us a function of these techniques — individual production, peer reviews, and standard implementation
\item other medical/health domains -- This worked for microbiome due to three reasons: nascent, personal, motivating 
\item nursing and midwife programs
\item auto-generating galileo for different tasks
\item teaching machines using procedures rather than declarative knowledge -- where the traning data is procedures
\item Automation -- can we learn from the data that we have collected? -- a classifier or something else? -- optimize backend -- 
\end{itemize}

% fig: a set that shows many approaches for generating hypotheses and evaluating theories
%-- and says these systems are all waiting to be built


%%%%%%
\section{Patterns} 
%"What are the research issues for advancing and crystallizing patterns in this area?" - srk
\subsection{Theoretical Questions}
\begin{itemize}
\item link between procedural and conceptual learning
\item does this link to transfer learning? -- do people improve what they learn?
\item what do we know about tacit knowledge — expert knowledge  --- where is this useful? what kind of knowledge is this?
\item transfer learning --  "We instead need to transfer knowledge from diverse prior experiences when trying to learn new tasks. Transfer learning"
\item How much can you get done with procedural learning -- its limits?
\item new ideas about interfaces like ????
situated cognition and distributed cognition -- pilots use rubrics and checklists etc...
\item counterfactual thinking -- looking at people’s responses and linking it with their microbiome, you can also predict their response to other unfilled questions and ask them?
%     see email threads with tricia
\end{itemize}

%%%%%%
\section{Methods} 
%"Reflecting on the methods you employed – system building, studies, etc. What worked best? What effort was wasted? What things would you do differently? And what would you recommend (or not) to others? (This might go adjacent challenges.)" - srk




%%%%%%
\section{Behavior} 
%"what do you want to know about human behavior? For example, we know that most behavioral interventions (like exercising more) are tough to stick. Speculate beyond your data about behavior. Generalize. Make wild guesses. And think about what next steps would help you get to those." - srk


%%%%%%
\section{More} 

\begin{itemize}
\item reuse techniques -- JIT learning seems one
\item which domains might this be useful in?	
-- auto-immune disorders
\item Cultural theories about poop — folk theories about poop 
(populate some details from online hunting)
\item Reproducibility — empirical pipeline 
— different levels of reproducing: just the experiment, teasing apart why it does or does not work
-- automatically converting to a paper — writing methods section with an eye on reproducibility 
\item collect folk theories: from people around the world by asking them to share theirs
— and then asking them to transcribe stuff (by reading one page each) — make it really fun
\item this general idea of making tacit knowledge explicit 
\item data -- how do i make sense to clinicians with my data -- life and times of data -- who produces it, where does it go, 
\item expertise questions --- expertise is a precious commodity; takes a lot to teach novices how to provide feedback etc  --PBD: make experts do something and then people to follow — step by step -- pamhinds -- zones of proximal development — using people rather than classes.. \\
 what is really the balance between expert-led work and citizen-led work
    corwdfunding
    citizen science
    more.. 
this balance between top-down and bottom-up work will identify our future
\end{itemize}




1. my immediate ideas 
2. how my research can help other areas too -- see all the people i saw on the market etc..

concerns with exp right now -- 1. minimal pairs 
2. placebo-controlled
3. double-blind



link to CBPR and community-led research
