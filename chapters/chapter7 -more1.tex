







%%%%%%%%%%%%%%%%%%%%%%%%%%%
\section{Behavior} 
%"what do you want to know about human behavior? For example, we know that most behavioral interventions (like exercising more) are tough to stick. Speculate beyond your data about behavior. Generalize. Make wild guesses. And think about what next steps would help you get to those." - srk

I would argue that these systems are only as exciting as the behavior that people demonstrate. 

We want to understand people's behavior as well as improve it. 
%condense all to like two lines
%"insights on who participates, what motivates them to participate, and how they participate”.
%todo - organize in terms of research questions

%link these to the topics discussed above
Another conceptual question is understanding the limits to procedural work. Based on procedural support as described above, it has limitations as well. This approach will struggle when when the concept being studied requires global awareness which is a hallmark of complex work: the sum is greater than the parts.  While procedural support can scale well, it might struggle for too “out there” ideas.

It's also important to not all complex work have this breakdown. For instance, consider collective activism. — there’s no set way to do this. How will the roles and procedural support look like for activities that do not have templated format like between-subjects scientific experimentation?  what happens when software does not have a strong model of the domain or a "high-level" recipe that can be improved.
%when do different elements of procedural guidance become important - templates for methods and iteration for discussion


\subsection{Does active doing lead to different behavior}
This general idea of making tacit knowledge explicit is super cool even for people themselves. This knwoledge comes from behavior in parts and also shapes behavior in part. With active doing, will poeple update their beliefs and their actions? A deeper question is how does offline doing translate to online doing? Does it, even?

Do people think in more evidence-rich ways in other domains?

So, basiclly, is there transfer from the offline to the online world, and from one domain to another?

Does it make people understand scientific work better?

Supporting complex work will require more than basic platform building but actually tying in to the motivation that people have.

\subsection{How to get more diverse people to participate?}

Complementing Global Data Collection with *Global* Distribution of Expertise is a worthwhile goal. This dissertation believes in a future of equal opportunity among people and equal skills. When people use such systems, two possible concerns come up: they provide a skewed population of the globe, and they perform little and drop out a lot. Plenty's been said about the latter, so i'll focus on the former.

Most Gut Instinct participants are from rich educated countries: 80\% of Docent questions were from people in the developed world and all 3 experimenters had advanced degrees. To brush away the difference in participation would be intellectually dishonest. More needs to be done to make the system meaningful for more people and to reach out to them. 

\textbf{Accept the multiplicative (not additive) effect of technology [??toyama]} - Rich country people are more likely to have more friends who are educated and willing to help out. System like Gut Instinct assume that people around the world have useful insights to share from their experiences. The results show this to be true. When people from educated, rich countries use these systems, they draw more attention to their concerns feeding into a feedback loop. Owing to how systems are linked with each others, such effects are more multiplicative rather than additive.  Missing people not represented on the platform across the world have different ideas and insights. we are missing out on those. People have the context, we want techniques to build expertise in people [raghuram rajan]

These are the concerns but how can we mitigate these issues.

%\textbf{Design phase}: Involve more relevant users in the design phase.

\textbf{Culturally-agnostic and -sensitive incentive design}
%see my angry locals email about this thing
A lot of our understanding is based on people in WEIRD countries [??WEIRD]. Since psychology studies have built upon fundamental ideas tested with Caucasian students in 1960s, it’s difficult to assume these apply to everyone, and especially with these results being questioned. People across different cultures might be motivated by different factors. 

Drawing on different cultural norms is important. LITW research has demonstrated that people across cultures have different flavor for “good” webpage design. TeachAIDS example where culturally-sensitive design changed outcomes: using locally tailored videos and other examples changed this.

There are two general themes to enable this: find the common things across cultures, and amplify uniquely for different populations. For the latter, basically, identify what’s the reality for people around the world and link it up to real stuff going on all around. Which of these techniques works better in engaging and improving contributions from prople? that itself could be interesting? Are we mostly similar or are we mostly different?

Some ways to find common grounds that cut across culture include: 
\begin{enumerate}
\item Self-promotion is another: Furthering one’s professional career and spreading one’s name is always a motivation. Similar to how Lil Wayne came from mixtapes [??], comedian Bill burr from podcasts, Xxx from twitter and blogs [??], Yyy from live-streaming[??]? In all these cases, people demonstrated their expertise by sharing their work in the open territory. While open science advocates pre-registering hypotheses and sharing insights, it still seems focused on experts; can we link such work with people’s real needs. will we find a scientist like this? doing their work and putting it in the open territory.

While other research like Crowd Research has looked at how other options like going to grad school is motivating, using “limited seats” opportunity to increase competition is not what i’m excited in. While creating an artificial sense of scarcity might be fine somewhat, it's not great. 
% what are other such research doing?

\item Temporal motivation and more: Motivation is not just cultural, it can also be temporal. E.g. When looking to draw broad participation on a topic like a disease, awareness months provide a natural time for people to be engaged in this, as demonstrated by the success fo ALS water bucket challenge. This gives a sense of collective work that trascends particular identities hopefully.

\item {Motivation to learn: Provide link to experts}: People like listening to experts (add Rob image). This might be even more pronounced in places where experts are considered distant.

\item{Extrinsic motivation in small parts}
While the willingness to do something might be based on personal meaningful nature of the task, actually doing it is more complex. People lack the time/remuneration/resources to learn new things and implement them in their lives. Might payment help? Combine intrinsic with extrinsic motivation provides one way. Pay people to start using the system, receive feedback, then have them continue using it [??personal conversation: jen mankoff]. 
\end{enumerate}

%todo-see east vs west common themese
Identify what works in local cultures and amplify them online. Some ways to engage specific audiences: 
\begin{enumerate}
\item Understand the complex economic and social factors. E.g. in societies that are strictly hierarchical and place experts way above novices, novices might be more concerned about challenging experts. Just because it has been put online does not mean it’s useful. 
\item Runtime - collaborating with on-ground experts: work with on-ground agencies, 3) Groups with greater motivation — patients
\item Altruism: While some might be motivated by self-interest and money, others might want to help out. 

\end{enumerate}

\textbf{Task and System design: Take an end-to-end view}

The blue book provides a great guide for what needs to be done. Here are some more specific ideas from my work.

\textbf{Initiation: will they feel they *can* and want to contribute?} With novel systems like Gut Instinct, people might not know or believe that they have much to contribute. Make this explicit to them — show examples, maybe identity-specific. Plenty of research demonstrates that when people find themselves in the right photo, they do more. As this Soylent post demonstrates, people might have different ideas about this platform. Communicating what the platform does is important. Do people feel this is meaningful for them?

\textbf{Make it easier for people to find what to contribute to and how - pre-registration might help} Diversity in roles: Technical interventions are scalable, social interventions are powerful. people’s interactions must be designed so they know what to do when. This is important due to a large diversity of both knowledge and participation  (dropout to leaders). 
%Take an iranian example and fix it

\textbf{continued use: }:  Do people find this useful now that they use it? How might the system better meet their needs?


\textbf{Finally, link the thing back}: Linking back the results with what they did — and inviting feedback 
%maybe fig for this thing


%todo- overall, link this with socio-technical gap
Not all social computing systems have the same affordances. It depends on the task being performed. Do standard tricks apply? Finally, be aware of the things that are lost with domain-specific social computing. standard online engagement tricks apply less to online scientific experimentation. People find online platforms engaging when they have social experience, receive feedback, and show their personality (personalize their profile, share photos, or share other info). Sadly, these activities as part of participating in an experiment can reduce scientific validity by nulling the independence of data assumption. What kinds of online interactions can be allowed (and tracked for conformity checks later) while still participating in an experiment? Future work can study that.

Many people around the world do not have access to education and other systematic resources. In the absence of functional institutions in many parts of the world (including huge swathes of “developed” countries), internet systems become even more important:  do they focus attention on questions of personal meaning that can improve one’s life or do they take attention and other resources away?

%Take for example the case of MOOCs and the stanford student who ranked 125 in the class after all other students from places around the world.

%%%%%%%%%%%%
\section{Implications}
%"If the world were to transform to rely heavily on your work, what broader technical and societal transformations would arise? How would your work scale? What would the social and technical challenges be? This might be a good place to connect your work to broader (relevant) writing on the role of technology in society"-srk

implications for social computing?
1. what it means for designing such systems and complex work
2. “how to eke out tacit knowledge from people"

\subsection{What does it mean to be an expert: doesn’t have to be solitary} 
What does it mean to be a scientist? While dictionary defines a scientist as “a person who is studying or has expert knowledge of one or more of the natural or physical sciences.”, it’s an overarching idea that includes a broad range of experts. 

For all their useful contributions, Gut Instinct users (exp designers, participants, hypotheses generators) cannot be considered scientists by me due to three reasons
1. people lack deeper, difficult to conceptualize skills:  even though people design and run an experiment, there’s many kinds of evaluation that’s missing — e.g. figuring out the right research question given what’ known in the domain — linking to existing research
2. people play a specific role - there’s a gradation and specific skillsets — people prefer specific tasks how do we teach people these and build upon these. But this also opens up opportunities:  how can we get people to do more? Drawing on the ideas of legitimate peripheral participation, getting people started on one small thing can help them with the other thing. However, is this always true? Typical demarcation of LPP has looked at different nature of tasks — edit, write, more; but most have still been in the realm of simple knowledge work: but why does this work and when does this break? This is a question for future study.
3. Science is terribly social [latour - use specific insight.] — those things are not baked in though and overarching views of science present it at this "aha moment by a lone genius" story. All the aha moments are so small and narrow that we cannot even imagine. Ideas are discussed over coffee, gained from talks, reading papers — it’s a really messy process that cannot be entirely translated to a template. People get ideas from their lifestyle and use them in research and vice-versa. Feynman won a nobel prize for a question he came across watching kids in a cafeteria. How do you create those serendipitous encounters?


Writing, like science, has this question: what does it mean to be a writer/scientist. What if someone else edits your paper or someone else actually runs the experiment? How do we allot credits in these cases. MSB has used credit allocation algorithms but where are they good and where do they fail. 

While history is ripe with visions of the lone warrior, reality is warmer to the idea of groups of people doing things. Complex work like science was considered a solitary endeavor, now its social. Some, like writing, are still considered solitary.

E.g. many scientists publish popular books in collaboration with ghost writers [?]; who is the writer in this case? This brings up the absurdity of using old school ideas for how things happen or at least how we increasingly observe things to happen. Unless a writer is handling everything pertaining to a book, Writing a book is a collaborative effort. There are reviewers who provide feedback, readers, publishers.. 

 -- young scientists need more time to master the growing body of knowledge that lies between them and the frontier of a field -- collaboration among scientists and novices

Also, roles come up normally/naturally in social computing — should these be linked to just participation on the platform or draw from other expertise things too?

Automatically converting to a paper — writing methods section with an eye on reproducibility
or using the procedural rules i have collected 

\textbf{What might be the role of experts in this new future?}

Expertise is a precious commodity; this dissertation did not use experts for feedback due to time needed and liability questions. However, there might be other ways in which experts can be involved. Experts have successful provided feedback [shepherd], built systems as designers, and more.. 

Similar to other domains like crowdfunding—where people (not VCs) fund startup ideas  and citizen science — balance of role division can change too. But it doesn’t necessarily need to be terrible for experts. Maybe experts can provide specific procedural rules of how they do things and people can follow that. One place to introduce this can be reproducibility of scientific research, using multimodal instructing using videos and audios and interfaces etc.. (btw, this is also an instance of how procedural rule might be useful).

Another way could be to involve citizens in expert work — one bottleneck that i’ve found is that experts need to recruit more people - find citizens to recruit others. 

People can also experiment and reproduce and provide inputs about mechanistic explanations (e.g. teasing apart why it does or does not work)

I see following ways for experts to contribute (see msb diss) — Quality control  and more.

But there are challenges too: Might experts nudge people into their own questions? or bias their doing by suggesting feedback?
    Involving experts while ensuring the bring useful knowledge but not the biases would be important. How do we do this? Experts might be more useful for structure/process oriented feedback while the domain knowledge could still be provided by standard resources. Might working with novices help experts uncover their blind spots?

%implications for the individual, the community, and machines -- challenges, potential ways, etc... -- atoms, bits, culture
   % link to CBPR and community-led research


\subsection{Folk theories are relevant in many domains}
%and learning interfaces
Where else might folk theories be relevant?

With our current times of fire and attack on truth, we are seeing an end ot the quest for an objective reality. Building a bridge between institutions and people’s lived experiences is important — we have processes for these but it’s not clear these are working. While numbers and data provide the illusion of objectivity, it’s not necessarily true: what if the research question is biased, what if measures aren’t correct. 
%https://www.snopes.com/news/2019/10/10/facts-and-data-arent-enough-to-combat-fake-news/

The key challenges include: 1) where are these relevant and how? 2) how to do is in a humane and efficient manner

A burgeoning field is developing around folk theories for narrative economics, cultural psychology, and more… 
%what do we know about tacit knowledge — expert knowledge  --- where is this useful? what kind of knowledge is this?

%how to do this
Polio took such a long time from results to policy --  how to get citizens to create such knowledge -- link to health world. We already know the difficulty of collecting data on the ground [akshay roongta thesis] -- Cultural effects in self organization of crowds?

Here are some ways to proceed.

First, Collect folk theories across cultures and then compare them and ask people for mechanistic explanations: from people around the world by asking them to share theirs; what might this system look like: -- what would it enjoy. Perhaps a metric of identifying plausible candidates would be cool -- would majority vote be the best? we don't know… For instance, think about how different cultures think differently of food — beef, pork, and more — what effect does it have on their nutrition en masse?

personal intuitions provide one set of hypo generation -- others could be folk theories, data tracked from devices, observations, playing around with data using Vega-lite; ways to evaluate could be different too -- data analysis




%%%%%%
%\section{More} 



\section{Shortcomings / Challenges / limitations}
%"What were the challenges of this work? For example, the difficulty of gathering longitudinal data b/c people are tied to their email client. How would you recommend others address these challenges?" - srk
Is this really scientific work though? To be able to publish in a top-tier journal and to inform policy, the experiments would need to satisfy domain-specific rules. This would inform more specific rules around manipulation (rather than self-adminstering the intervention), use of placebo, and so on. We also find these to be useful avenues for further social computing use: enabling family members to administer the intervention based on reminders, or even enabling them to randomize that order... 
	Also minimal pairs and double blind -- how would you implement it socially without bias -- this trade-off between social trust and bias can be interesting -- wll it be like x or like y


%corporations have so many resources -- how do we do this?
%	engagement metrics implemneted by people